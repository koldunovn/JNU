\documentclass{beamer}

\usepackage{graphicx}
\usepackage[absolute,overlay]{textpos}
\usepackage{listings}
\usepackage{color}
\usepackage{textcomp}
\definecolor{listinggray}{gray}{0.9}
\definecolor{lbcolor}{rgb}{0.9,0.9,0.9}
\lstset{
	backgroundcolor=\color{lbcolor},
	tabsize=4,
	rulecolor=,
	language=matlab,
        basicstyle=\tiny,
        upquote=true,
%         aboveskip={1.5\baselineskip},
        columns=fixed,
        showstringspaces=false,
        extendedchars=true,
        breaklines=true,
        prebreak = \raisebox{0ex}[0ex][0ex]{\ensuremath{\hookleftarrow}},
        %frame=single,
        showtabs=false,
        showspaces=false,
        showstringspaces=false,
        identifierstyle=\ttfamily,
        keywordstyle=\color[rgb]{0,0,1},
        commentstyle=\color[rgb]{0.133,0.545,0.133},
        stringstyle=\color[rgb]{0.627,0.126,0.941},
}


% \usepackage{default}
\usetheme{Copenhagen}
\newcommand{\blankline}{\quad\pagebreak[2]}
\title{CLACINDIA Workshop \\
Lab Session\\
       Part 2:  Programming languages}
\author{Nikolay Koldunov}

\date{JNU\\
       2014}

\pgfdeclareimage[height=0.8cm]{logo}{logos}
\setlength{\TPHorizModule}{1mm}
\setlength{\TPVertModule}{1mm}
\newcommand{\MyLogo}{%
\begin{textblock}{14}(85.0,81.0)
  \pgfuseimage{logo}
\end{textblock}
}


\begin{document}


\begin{frame}
\titlepage
\MyLogo
\end{frame}

%%%%%%%%%%%%%%%%%%%%%%%%%%%%%%
%%%%%%%%%%%%%%%%%%%%%%%%%%%%%%%

\begin{frame}[fragile]
\frametitle{Programming languages}

A \textbf{Programming language} is an artificial language designed to express computations that can be performed by a machine, particularly a computer. Programming languages can be used to create programs that control the behavior of a machine.
\begin{center}
\includegraphics[width=0.5\textwidth,angle=00]{languages.jpg}
\end{center}
\end{frame}
%%%%%%%%%%%%%%%%%%%%%%%%%%%%%%
%%%%%%%%%%%%%%%%%%%%%%%%%%%%%%%

\begin{frame}[fragile]
\frametitle{``Hello world!'' program}

\begin{block}{MATLAB}
  \begin{lstlisting}
disp('hello world')
  \end{lstlisting}
\end{block}

\pause

\begin{block}{FORTRAN 90}
  \begin{lstlisting}
PROGRAM HelloWorld
   WRITE(*,*)  "Hello World!"
END PROGRAM
  \end{lstlisting}
\end{block}

\pause

\begin{block}{C++}
  \begin{lstlisting}
#include <iostream.h>
main()
{
    cout << "Hello World!" << endl;
    return 0;
}
  \end{lstlisting}
\end{block}


\end{frame}

%%%%%%%%%%%%%%%%%%%%%%%%%%%%%%%%%%%%%%%%%%%%%%%%
%%%%%%%%%%%%%%%%%%%%%%%%%%%%%%%%%%%%%%%%%%%%%%%%%%

\begin{frame}[fragile]
\frametitle{``Hello world!'' program}

\begin{block}{BrainF**k}
  \begin{lstlisting}
++++++++++[>+++++++>++++++++++>+++<<<-]>++.>+.+++++++
..+++.>++.<<+++++++++++++++.>.+++.------.--------.>+.

  \end{lstlisting}
\end{block}

\pause

\begin{block}{Bit}
  \begin{lstlisting}
LINENUMBERZEROCODEPRINTZEROGOTOONELINENUMBERONECODEPRINTONEGOTOONEZEROLINENUMBE
RONEZEROCODEPRINTZEROGOTOONEONELINENUMBERONEONECODEPRINTZEROGOTOONEZEROZEROLINE
NUMBERONEZEROZEROCODEPRINTONEGOTOONEZEROONELINENUMBERONEZEROONECODEPRINTZEROGOT
OONEONEZEROLINENUMBERONEONEZEROCODEPRINTZEROGOTOONEONEONELINENUMBERONEONEONECOD
EPRINTZEROGOTOONEZEROZEROZEROLINENUMBERONEZEROZEROZEROCODEPRINTZEROGOTOONEZEROZ
EROONELINENUMBERONEZEROZEROONECODEPRINTONEGOTOONEZEROONEZEROLINENUMBERONEZEROON

  \end{lstlisting}
\end{block}


\end{frame}

%%%%%%%%%%%%%%%%%%%%%%%%%%%%%%%%%%%%%%%%%%%%%%%%
%%%%%%%%%%%%%%%%%%%%%%%%%%%%%%%%%%%%%%%%%%%%%%%%%%

\begin{frame}[fragile]
\frametitle{Popular programming languages for the Earth Sciences}

\begin{itemize}
 \item Fortran (Mathematical \textbf{For}mula \textbf{Tran}slating System)
\pause
 \item MATLAB (\textbf{Mat}rix \textbf{Lab}oratory) 
\pause 
\item Python 
\pause 
\item IDL (Interactive Data Language)
\pause 
\item bash, csh
\pause 
\item NCL (NCAR Command Language)

\end{itemize}



\end{frame}
%%%%%%%%%%%%%%%%%%%%%%%%%%%%%%%%%%%%%%%%%%%%%%%%
%%%%%%%%%%%%%%%%%%%%%%%%%%%%%%%%%%%%%%%%%%%%%%%%%%



%%%%%%%%%%%%%%%%%%%%%%%%%%%%%%%%%%%%%%%%%%%%%%%%
%%%%%%%%%%%%%%%%%%%%%%%%%%%%%%%%%%%%%%%%%%%%%%%%%%



\end{document}